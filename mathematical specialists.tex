\documentclass[a4paper,10pt]{report}

\usepackage[utf8]{inputenc}
\usepackage{cancel}
\usepackage[margin=2cm]{geometry}
\usepackage{amsmath}
\usepackage{amssymb}
\usepackage{apacite}

% Title Page
\title{Mathematical Specialists}
\author{Lachlan Takumi Ikeguchi}


\begin{document}
	\maketitle
	\tableofcontents

	\begin{abstract}
		This document was written to be used as a summary to help revise the content covered mathematical specialists.  For any inquiries, contact lachlanprivate@duck.com or through the discord server: https://discord.gg/6P8rddkXFr
    \end{abstract}

    \section{Important symbols}
		\begin{center}
			\begin{tabular}{l|lp{6cm}}
				Symbol & Mathematical definition                                                  & Simple definition \\ \hline
				$\cup$ & Union                                                                    & $A$ \emph{and} $B$, or think as in 'add'.\\
				$\cap$ & Intersection                                                             & $A$ \emph{or} $B$, or think as in 'multiply'.\\
				$n!$   & $n \times (n-1) \times (n-2) \times (n-3)... \times 3 \times 2 \times 1$ & The product of integers between the given value and 1.\\
				$^nP_r$& $\frac{n!}{(n-1)!}$ 													  & The number of combinations there are of length $r$ from a group of length $n$ where the order matters.\\
				$^nC_r$& $\frac{\frac{n!}{(n-1)!}}{r!}$                                           & The number of combinations there are of length $r$ from a group of length $n$ where the order does not matter.\\
			\end{tabular}
		\end{center}

	\pagebreak

	\section{Addition principle}
		If there are $n$ ways of performing operation $A$, and $m$ ways of performing operation $B$, there are $n + m$ ways of performing operation $A \cap B$.\\

		As in, let there be 2 ways to perform task $A$, and 3 ways to perform task $B$.  If there is an option to perform $A$ \emph{or} $B$, there is a total of $2 + 3 = 5$ ways to perform an operation.


	\section{Multiplication principle}
		If there are $n$ ways of performing operation $A$, and $m$ ways of performing operation $B$, there are $n \times m$ ways of performing operation $A \cup B$.\\

		As in, let there be 4 ways to perform task $A$, and 5 ways to perform task $B$.  With considering 1 way to perform $A$, there are 5 ways to perform $B$, repeat the process through the number of ways to perform $A$.
		\begin{center}
			\begin{tabular}{l|lllll}
				   & $B$, 1 & $B$, 2 & $B$, 3 & $B$, 4 & $B$, 5\\ \hline
			$A$, 1 & (1, 1) & (1, 2) & (1, 3) & (1, 4) & (1, 5)\\
			$A$, 2 & (2, 1) & (2, 2) & (2, 3) & (2, 4) & (2, 5)\\
			$A$, 3 & (3, 1) & (3, 2) & (3, 3) & (3, 4) & (3, 5)\\
			$A$, 4 & (4, 1) & (4, 2) & (4, 3) & (4, 4) & (4, 5)\\
			\end{tabular}
		\end{center}
		Where there are $4 \times 5 = 20$ ways to perform the operations.


	\section{Factorials}
		Factorials is the result of multiplying all of the integers between the given integer and 1.  This is given the notation and formula:
		$$
			n! = n \times (n-1) \times (n-2) \times (n-3)... \times 3 \times 2 \times 1
		$$
		Example:\\
		\begin{align*}
			5! & = 5 \times 4 \times 3 \times 2 \times 1 \\
			   & = 120
		\end{align*}


		\subsection{Dividing factorials}
			When given some factorial over another factorial in such cases as:
			$$\frac{4!}{2!}$$
			$$\frac{4 \times 3 \times 2 \times 1}{2 \times 1}$$
			$$\frac{4 \times 3 \times \cancel{2 \times 1}}{\cancel{2 \times 1}}$$
			$$4 \times 3$$
			$$12$$

		\subsection{Special cases}
			$$0! = 1$$


	\section{Permutations}
		Permutations is the number of ways of choosing $r$ things from $n$ distinct things where \emph{order matters}.  This is given the notation and formula:
		$$
			^nP_r = \frac{n!}{(n-r)!}
		$$
		Example:\\
		\begin{align*}
			^6P_4 & = \frac{6!}{(6 - 4)!} \\
			      & = \frac{6!}{2!} \\
			      & = \frac{6 \times 5 \times 4 \times 3 \times 2 \times 1}{2 \times 1} \\
			      & = \frac{6 \times 5 \times 4 \times 3 \times \cancel{2 \times 1}}{\cancel{2 \times 1}} \\
			      & = 6 \times 5 \times 4 \times 3 \\
			      & = 360
		\end{align*}

		\subsection{Permutations in a circle}
			In cases where the positions being concerned is a circle such as the permutations of a circular seating arrangement, if the standard permutations formula is applied, there are several over-counting of the like arrangement in a different perspective.  In these cases apply the formula:
			$$
				\frac{^nP_r}{r} = \frac{\frac{n!}{(n-r)!}}{r}
			$$

		\subsection{Like objects repetitions}
			The number of ways of arranging $n$ objects made up of indistinguishable objects, $n_1$ in the first group, $n_2$ in the second group and so on, is:
			$$
				\frac{n!}{n_1! n_2! n_3!... n_r!}
			$$
			Example:\\
			\begin{center}
				Find the number of permutations of the letters in the world WOOLLOOMOOLOO.\\
				There are 8 'O's and 3 'L's\\
				Permutations:
				$$\frac{13!}{8!3!}$$
				$$\frac{13 \times 12 \times 11 \times 10 \times 9 \times 8 \times 7 \times 6 \times 5 \times 4 \times 3 \times 2 \times 1}{(8 \times 7 \times 6 \times 5 \times 4 \times 3 \times 2 \times 1) \times (3 \times 2 \times 1)}$$
				$$\frac{13 \times 12 \times 11 \times 10 \times 9 \cancel{\times 8 \times 7 \times 6 \times 5 \times 4 \times 3 \times 2 \times 1}}{\cancel{(8 \times 7 \times 6 \times 5 \times 4 \times 3 \times 2 \times 1)} \times (3 \times 2 \times 1)}$$
				$$\frac{13 \times 12 \times 11 \times 10 \times 9}{3 \times 2}$$
				$$\frac{13 \times \cancelto{6 \cancel{\times 2}}{12} \times 11 \times 10 \times \cancelto{3 \cancel{\times 3}}{9}}{\cancel{3 \times 2}}$$
				$$13 \times 6 \times 11 \times 10 \times 3$$
				$$25,740$$
			\end{center}

		\subsection{Restrictions}
			When considering restrictions, deal with the restrictions first.\\
			Example:\\
			\begin{center}
				Find the number of arrangements of the letters of the word DARWIN beginning and ending with a vowel.\\
				Number of letters = 6\\
				Number of positions = 6\\
				Beginning and end must be a vowel, so the available positions are decreased by 2: number of positions - 2 = 4\\
				Since 2 vowels must be used:  number of letters - 2 = 4\\
				Permutations:
				$$^4P_4$$
				$$4!$$
				$$4 \times 3 \times 2 \times 1$$
				$$24$$
			\end{center}

		\subsection{Grouped items}
			When items are grouped together, treat each group as a single object.  Find the number of arrangements of the groups, then multiply by the number of arrangement within each group.\\
			Example:\\
			\begin{center}
				Find the number of arrangement of the letters of the word EQUALS if the vowels are kept together.\\
				Number of vowels = 3\\
				Number of letters = 6\\
				Number of positions = 6\\
				As the 3 vowels are grouped together, the number of positions decreases:\\ number of positions - 3\textsubscript{for the members of the group} + 1\textsubscript{for the group} = 4\\
				Permutations within vowel group:
				$$^3P_3 = 3! = 3 \times 2 = 6$$
				Permutations of all positions:
				$$^4P_4 = 4! = 4 \times 3 \times 2 = 24$$
				Multiply together:
				$$6 \times 24 = 144$$
			\end{center}

		\subsection{Special cases}
			$$^nP_n = n!$$
			$$^nP_0 = 1$$


	\section{Combinations}
		Combinations is the number of ways of choosing or selecting $r$ objects from $n$ distinct objects where \emph{order does not matter}.  This is given the notation and formula:
		$$^nC_r = \frac{^nP_r}{r!}$$
		$$^nC_r = \frac{\frac{n!}{(n-r)!}}{r!}$$
		Example:\\
		\begin{align*}
			^4C_2 & = \frac{^4P_2}{2!} \\
			      & = \frac{\frac{4!}{(4 - 2)!}}{2!} \\
			      & = \frac{\frac{4!}{2!}}{2!} \\
			      & = \frac{\frac{4 \times 3 \times 2}{2}}{2} \\
			      & = \frac{\frac{4 \times 3 \cancel{\times 2}}{\cancel{2}}}{2} \\
			      & = \frac{4 \times 3}{2} \\
			      & = \frac{12}{2} \\
			      & = 6 \\
		\end{align*}


	\section{Pascal's triangle}
		The Pascal's triangle is a pattern formed by adding the top 2 adjacent numbers and a 1 is placed on either side of the bottom row to resemble a triangle:
		\begin{center}
			\begin{tabular}{rccccccccccccccccccccc}
				$n=0$:&    &    &    &    &    &    &    &    &    &    &  1\\\noalign{\smallskip\smallskip}
				$n=1$:&    &    &    &    &    &    &    &    &    &  1 &    &  1\\\noalign{\smallskip\smallskip}
				$n=2$:&    &    &    &    &    &    &    &    &  1 &    &  2 &    &  1\\\noalign{\smallskip\smallskip}
				$n=3$:&    &    &    &    &    &    &    &  1 &    &  3 &    &  3 &    &  1\\\noalign{\smallskip\smallskip}
				$n=4$:&    &    &    &    &    &    &  1 &    &  4 &    &  6 &    &  4 &    &  1\\\noalign{\smallskip\smallskip}
				$n=5$:&    &    &    &    &    &  1 &    &  5 &    & 10 &    & 10 &    &  5 &    &  1\\\noalign{\smallskip\smallskip}
				$n=6$:&    &    &    &    &  1 &    &  6 &    & 15 &    & 20 &    & 15 &    &  6 &    &  1\\\noalign{\smallskip\smallskip}
				$n=7$:&    &    &    &  1 &    &  7 &    & 21 &    & 35 &    & 35 &    & 21 &    &  7 &    &  1\\\noalign{\smallskip\smallskip}
				$n=8$:&    &    &  1 &    &  8 &    & 28 &    & 56 &    & 70 &    & 56 &    & 28 &    &  8 &    &  1\\\noalign{\smallskip\smallskip}
				$n=9$:&    &  1 &    &  9 &    & 36 &    & 84 &    & 126 &    & 126 &    & 84 &    & 36 &    &  9 &    &  1\\\noalign{\smallskip\smallskip}
				$n=10$:&  1 &    & 10 &    & 45 &    & 120 &    & 210 &    & 252 &    & 210 &    & 120 &    & 45 &    & 10 &    &  1\\\noalign{\smallskip\smallskip}
			\end{tabular}
		\end{center}

		Each element in the Pascal's triangle can be used to calculate combinations, hence, the triangle can be written using Combinations notation ($^nC_r$):
		\begin{center}
			\begin{tabular}{rcccccccccccccccc}
				$n=0$:&    &    &    &    &    &    &    &    &    &    &  $^0C_0$\\\noalign{\smallskip\smallskip}
				$n=1$:&    &    &    &    &    &    &    &    &    &  $^1C_0$ &    &  $^1C_1$\\\noalign{\smallskip\smallskip}
				$n=2$:&    &    &    &    &    &    &    &    &  $^2C_0$ &    &  $^2C_1$ &    &  $^2C_2$\\\noalign{\smallskip\smallskip}
				$n=3$:&    &    &    &    &    &    &    &  $^3C_0$ &    &  $^3C_1$ &    &  $^3C_2$ &    &  $^3C_3$\\\noalign{\smallskip\smallskip}
				$n=4$:&    &    &    &    &    &    &  $^4C_0$ &    &  $^4C_1$ &    &  $^4C_2$ &    &  $^4C_3$ &    &  $^4C_4$\\\noalign{\smallskip\smallskip}
				$n=5$:&    &    &    &    &    &  $^5C_0$ &    &  $^5C_1$ &    & $^5C_2$ &    & $^5C_3$ &    &  $^5C_4$ &    &  $^5C_5$\\\noalign{\smallskip\smallskip}
			\end{tabular}
		\end{center}

		Pascal's triangle shows that the $r^\text{th}$ element of the $n^\text{th}$ row of Pascal's triangle is given by $^nC_r$.  It is assumed that the 1 at the beginning of each row is the $0^\text{th}$ element.  This gives the \emph{Pascal's identity}:
		$$^nC_r = ^{n-1}C_{r-1} + ^{n-1}C_r \text{ for } 0 < r < n$$

		The Pascal's triangle can be extended to the binomial theorem, where the rule for expanding an expression such as $(a + b)^n$.  Where:
		\begin{align*}
			(a + b)^ 0 &= 1\\
			(a + b)^ 1 &= a^1 + ^1\\
			(a + b)^ 2 &= a^2 + 2a^1b^1 + b^2\\
			(a + b)^ 3 &= a^3 + 3a^2b^1 + 3a^1b^2 + b^3\\
			(a + b)^ 4 &= a^4 + 4a^3b\\
			(a + b)^ 5 &= \\
		\end{align*}


	\bibliography{bibliography}
	\bibliographystyle{apacite}
\end{document}
